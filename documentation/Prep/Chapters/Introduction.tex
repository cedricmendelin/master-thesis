\chapter{Introduction}
\label{sec:introduction}

Inverse problems aim to estimate an original signal that went through a system, based on the output signal observation.
Machine learning (ML) is a tool to model and solve such inverse problems.
They are widely used throughout different science directions, such as ML,
signal processing, computer vision, natural language processing and many more.

In recent years, graphs got a lot of attention in ML and are one of the most promising research areas.
Graphs are a well suited data structure, simple but with high expressiveness. 
For some specific scenarios, ordinary ML algorithm fail but Graph ML approaches have great success, e.g dimensionality reduction for high-dimensional data.
Data can be in a graph structure already, like social networks, or they can be constructed for arbitrary datasets.


Cryo-electron microscopy (cryo-EM), where molecules are imaged in an electron microscope,
gained a lot of attention in recent years. 
Due to ground-breaking improvements regarding hardware and data processing, the field of research
has highly improved. In 2017, pioneers in the field of cryo-EM got the 
Nobel Prize in Chemistry\footnote{https://www.nobelprize.org/prizes/chemistry/2017/press-release/}.
Today, using cryo-EM many molecular structures can be observed with near-atomic resolution.
The big challenge with cryo-EM is enormous noise. 

\bigskip

The following report resulted from Master Thesis Preparation. During the six week project, 
goal is to familiarize with research area, build up mathematical foundation 
and define project content as well as a project plan.
The report is structured the following:

In chapter~\ref{sec:imaging}, the main motivation of the Master Thesis is given. The two imaging methods
computed tomography and cryo-EM are introduced and an abstract model.
Chapter~\ref{sec:graphDenoising} is dedicated to graphs, were the connection of graphs to computed tomography and 
cryo-EM is established. Further, the problem of "Graph Denoising" is defined and methods like graph
construction and Graph Laplacian are introduced.
Finally, in chapter~\ref{sec:projectOverview} project content is shortly concluded, work packages are defined and project schedule is given.
Throughout the report, red boxes are used to denote important statements
regarding final Master Thesis project.
