\chapter{Introduction}
\label{sec:introduction}

Inverse problems aim to estimate an original signal that went through a system, 
based on a potentially noisy output signal observations.
They are widely used throughout different science directions, such as Machine Learning,
signal processing, computer vision, natural language processing and others.
ML is a tool to model and solve such inverse problems.


\bigskip

In recent years, graphs got a lot of attention in ML and are one of the most promising research areas.
Graphs are a well suited data structure, simple but with high expressiveness. 
Especially for data where single data point tend to have a relation to other data points, graphs with its nodes and vertices are the perfect tool
to capture these relationships. 
Data can be in a graph structure already, like social networks, or they can be artificially constructed for arbitrary datasets.
Moreover, for some scenarios, ordinary ML algorithms fail, but Graph ML approaches have great success, e.g. dimensionality reduction for high-dimensional data.


\bigskip

Cryo-electron microscopy (cryo-EM), where molecules are imaged in an electron microscope,
is a molecular imaging method and gained a lot of attention in recent years. 
Due to ground-breaking improvements regarding hardware and data processing, the field of research
has highly improved. In 2017, pioneers in the field of cryo-EM got the 
Nobel Prize in Chemistry\footnote{https://www.nobelprize.org/prizes/chemistry/2017/press-release/}.
Today, using cryo-EM, molecular structures can be observed with near-atomic resolution.
The big challenge with cryo-EM is enormous noise and unknown observation angles.

Computed tomography (CT) is a similar problem than cryo-EM, but slightly easier
as the problem is in two dimensions and observation angles are known.
The overall goal of this Thesis is, to introduce an algorithm which works with CT, but 
can conceptually be extended to work in 3D, and therefore for cryo-EM.

\bigskip

As a result of this Thesis, a Graph Neural Network (GNN) architecture is proposed, which is called \textit{GAT-Denoiser}.
GAT-Denoiser aims to denoise noisy observation from CT to improve overall reconstruction quality.
In the GNN architecture, convolution and Graph Attention Network (GAT) is used to denoise observations.
Further, an end-to-end learning approach with U-Net is used to further improve reconstruction quality, 
where a pre-trained U-Net is jointly optimized further during GAT-Denoiser training.

\textbf{TODO: Refer to final results}

On the LoDoPaB-CT dataset, GAT-Denoiser outperformed baseline algorithms, such as BM3D, and shows great results.
From given observation signal-to-noise-ratio (SNR) of -15 dB, it realizes an average reconstruction SNR of XX dB.

\bigskip

The Thesis is structured the following: 


In Chapter~\ref{sec:imaging} \textit{\nameref{sec:imaging}}, the two molecular imaging methods
computed tomography and cryo-EM are introduced as well as a mathematical abstraction for observation and reconstruction.
Further, Chapter~\ref{sec:graphFoundations} \textit{\nameref{sec:graphFoundations}} is dedicated to graphs, 
where connection from graphs to computed tomography and cryo-EM is established. 
Moreover, the problem of "Graph Denoising" is defined and methods like graph construction and Graph Laplacian are introduced.
The main concept of GAT-Denoiser is described in Chapter~\ref{sec:contribution} \textit{\nameref{sec:contribution}}
and results are presented in Chapter~\ref{sec:results} \textit{\nameref{sec:results}}.
Finally, conclusion and future work are presented in Chapter~\ref{sec:Conclusion} \textit{\nameref{sec:Conclusion}}.

