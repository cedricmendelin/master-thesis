\chapter{Molecular Imaging Methods}
\label{sec:imaging}

This chapter introduces two molecular imaging methods, \textit{Computed Tomography} (CT) and 
\textit{Cryo-Electron Microscopy} (cryo-EM). 
Further, their observation model is defined in a mathematic way and reconstruction is presented.
Application of cryo-EM is a major motivation for this Thesis, 
as the problem is not easy to solve due to dealing with enormous noise and unknown observation angles.
Moreover, cryo-EM can be seen as a problem in 3D, as original object is in 3D.
CT is similar to cryo-EM, but reconstruction is slightly simpler, since it is potentially in 2D and 
observation angles are known.
That's why it is well suited as a first step towards a cryo-EM algorithm.



\section{Computed tomography}
CT is a well established molecular imaging method.
Using X-ray source, fan shaped beams are produced which scan the biological sample.
Through scanning over straight lines many observations are collected, 
and the biological sample can be reconstructed.\footnote{For further details \cite{computedTomography}}

\paragraph{2D tomographic observation:}

Mathematically, CT observations are defined as follows:

\begin{equation}
    \label{eq:2Dreconstruction}
    \begin{aligned}
        y_i[j] &= p_i + \eta_i[j], & \text{ with } 1 \leq i \leq N \text{ and } 1 \leq j \leq M \\
               &= R(x, \theta_i, s_j) + \eta_i[j], & \text{ with } 1 \leq i \leq N \text{ and } 1 \leq j \leq M
    \end{aligned}
\end{equation}

with
\begin{itemize}
    \item $x \in L^2(\Omega)$ : original object with $\Omega \subset \mathbb{R}^2 $ and $L^2$ Lebesgue space
    \item $R(\cdot; \theta_i, s_j): L^2(\Omega) \to \mathbb{R}^M , x \mapsto R(x; \theta_i, s_j)$ : Radon Transform\footnote{For additional information \cite{radonTransform}} 
        with, $\theta_i \in \mathbb{S}^1$ : observation angle and $s_j \in \mathbb{R}$ : sampling point
    \item $\eta_i \in \mathbb{R}^M$ : i.i.d Gaussian noise with $\eta_i[j] \sim \mathcal{N}(0,\sigma^2) \in \mathbb{R}$
\end{itemize}


\paragraph{Observation illustration:}

Output of Radon Transform is called a \textit{sinogram}, which is the CT observation.
In Figure~\ref{fig:phantom} the Shepp-Logan phantom is illustrated.
It is often used as an image for simulating a brain CT and in refers to the biological sample $x$. 
Further, in Figure~\ref{fig:phantom_sinogram} and Figure~\ref{fig:phantom_sinogram_noisy} 
observation sinograms can be seen with and without noise respectively. 
$\theta$ and $s$ must be defined to apply Radon Transform.
For this example, $\theta \in \mathbb{R}^{500}$ was evenly spaced
between $[0, 2 \pi]$ and $dim(s) = 400$. 
Thus, $p \in \mathbb{R}^{500 \times 400}$ and can be plotted as image with resolution 500x400. 
Further, noise was added to reach an SNR of 10 dB.


\begin{figure}[H]
    \captionsetup[subfigure]{justification=centering}
    \centering
    \begin{subfigure}[t]{0.3\textwidth}
        \includegraphics[width=\textwidth]{phantom.png}
        \caption{Shepp-Logan phantom : $x$}
        \label{fig:phantom}
    \end{subfigure}\hfill
    \begin{subfigure}[t]{0.3\textwidth}
      \includegraphics[width=\textwidth]{phantom_sino.png}
      \caption{Clean sinogram: $R(x, \theta, s)$}
      \label{fig:phantom_sinogram}
    \end{subfigure}\hfill
    \begin{subfigure}[t]{0.3\textwidth}
      \includegraphics[width=\textwidth]{phantom_sino_noisy_snr10.png}
      \caption{noisy sinogram: $R(x, \theta, s) + \eta = y + \eta$}
      \label{fig:phantom_sinogram_noisy}
    \end{subfigure}
    \caption{Shepp-Logan phantom and corresponding sinograms.}
    \label{fig:phantom_and_sinos}
  \end{figure}


\paragraph{Tomography reconstruction:}

Tomographic reconstruction is a popular inverse problem. 
The aim is to reconstruct a biological sample based on observations.
When original object is in 2D, observations are available in 1D. 
The reconstruction is possible for 3D objects as well, where observations are in 2D\footnote{For additional information \cite{tomographicReconstruction}}.
In this Thesis I focus on the 2D case of CT, which is called \textit{classical tomography }


\subparagraph{Filter Backprojection:}
Filter Backprojection (FBP)~\cite{tomographicReconstruction} is a reconstruction method used in classical tomography.
Until recently, it was the primary method for reconstruction as it allows to inverse the Radon Transform and 
enables reconstruction of the original object $x$.

FBP can be defined as:

\begin{equation}
    \label{eq:fbp}
    \textit{FBP}(\cdot; \theta_i, s_j) : \mathbb{R}^M \to L^2(\Omega), y \mapsto \textit{FBP}(y; \theta_i, s_j)
\end{equation}

Where $\theta$ are the projection angles and $s$ are the sampling points.
The algorithm fails when working with highly noisy data~\cite{cryoEmMath2}, as it is not possible to draw meaningful connections anymore, noise
is dominating information in the data.

Therefore, alternatives for the high noise domain are studied in recent year.
Lately, neural network approaches emerged, which further process the output of FBP to increase reconstruction quality.
This is the approach I will follow in this Thesis. \citet{ct-reconstruction-comparison} compared different 
Deep-Learning reconstruction methods for CT. 

\subparagraph{U-Net}
Today's state-of-the-art reconstruction algorithms are Deep-Learning based.
U-Net~\cite{unet-tomography}, a convolution neural network approach, performed
much better compared to FBP in the comparison of \citet{ct-reconstruction-comparison}.
Therefore, it is an interesting baseline and additionally can be combined with other ideas.

In Figure~\ref{fig:phantom_fbps} reconstructions from Shepp-Logan phantom are presented.
Noise in $y$ is defined to reach an \snry of 10 dB.

The quality of noisy reconstruction is rather low, some important details are missing, and the noise dominates reconstruction.
If more noise is present in $y$, reconstruction will be of even lower quality.
Further, Figure~\ref{fig:fbp_unet_phantom_noisy} 
shows reconstruction with U-Net where some details are missing as well, although overall noise is drastically decreased.


\begin{figure}[H]
    \captionsetup[subfigure]{justification=centering}
    \centering
    \begin{subfigure}[t]{0.3\textwidth}
        \includegraphics[width=\textwidth]{fbp_phantom_clean.png}
        \caption{Clean reconstruction: $\textit{FBP}(p, \theta, s)$}
        \label{fig:fbp_phantom}
    \end{subfigure}\hfill
    \begin{subfigure}[t]{0.3\textwidth}
      \includegraphics[width=\textwidth]{fbp_phantom_snr_10.png}
      \caption{Noisy reconstruction: $\textit{FBP}(y, \theta, s)$}
      \label{fig:fbp_phantom_noisy}
    \end{subfigure}\hfill
    \begin{subfigure}[t]{0.3\textwidth}
      \includegraphics[width=\textwidth]{fbp_unet_phantom_snr_10.png}
      \caption{Noisy reconstruction: $\textit{UNet}(\textit{FBP}(y, \theta, s))$}
      \label{fig:fbp_unet_phantom_noisy}
    \end{subfigure}
    \caption{Shepp-Logan reconstructions with \snry 10 dB.}
    \label{fig:phantom_fbps}
  \end{figure}


\section{Cryo-EM}
Cryo-EM is another molecular imaging method, that enables the view of molecules in near-atomic resolution.
In this Thesis, for simplicity, only single-particle cryo-EM~\cite{singleParticleCryoEm} is considered.
When writing about cryo-EM it always refers to single-particle cryo-EM.

During the imaging process molecules are frozen in a thin layer of ice, where they are randomly oriented and positioned. 
Random orientation and positioning makes reconstruction challenging, 
but freezing allows observation of their conformation in a stable state where molecules are not moving.
With an electron microscope, 2D tomographic projection images of molecules are observed,
which are called \textit{micrograph}. 
Frozen molecules are fragile and electron microscope needs to work with
very low power (electron dose), resulting in highly noisy observations. The resulting SNR
is typically smaller than 0 dB, which indicates that there is more noise than signal \cite{cryoEmMath2}.

In addition, observed molecules are not equal in the sense that there are some structural varieties between
molecules (isotopes). While observing the same molecule in ice many times, single observations could be from different isotopes.


\textbf{TODO: some inconsistency with M and the number of observation. 
I think Delta should be Omega power M , and then indices (j,k) are taken from Delta.}

\paragraph{3D cryo-EM observation:}
Mathematically, observation is defined as follows:
\begin{equation}
    \label{eq:cryoEmSimple}
    \begin{aligned}
        y_i &= p_i + \eta_i, &\text{ with } 1 \leq i \leq N,\\
        y_i &= \Pi_z  (\; Rot (\;x; \theta_i )) + \eta_i, &\text{ with } 1 \leq i \leq N,    
    \end{aligned}
\end{equation}

where 
\begin{itemize}
    \item $x \in L^2(\Omega)$: original object with $\Omega \subset \mathbb{R}^3 $ and $L^2$: Lebesgue space
    \item $\Pi_z : L^2(\Omega) \to L^2(\tilde{\Omega}), x \mapsto  \int x(\cdot,\cdot,z) dz$: z-axis projection operator,
          with $\tilde{\Omega} \subset \mathbb{R}^2$
    \item $\theta_i = [\theta_i^{(1)}, \theta_i^{(2)}, \theta_i^{(3)} ] $: 3D rotation matrix with $ \theta_i^{(1)}, \theta_i^{(2)}, \theta_i^{(3)} \in \mathbb{R}$ and \\
          $R_{\theta_i} =  R_{e_x} (\theta_i^{(1)}) R_{e_y} (\theta_i^{(2)}) R_{e_z} (\theta_i^{(3)}) = [R^1_{\theta_i}, R^2_{\theta_i}, R^3_{\theta_i}] \in SO(3)$ 
          \footnote{(for further details see \ref{app:3DrotationMatrix})}
          
    \item $\textit{Rot} : L^2(\Omega) \to L^2(\Omega), \textit{Rot}(x, \theta_i) = \left((x_1,x_2,x_3) \mapsto x( x_1R^1_{\theta_i}, x_2R^2_{\theta_i}, x_3R^3_{\theta_i})\right)$: rotation operator
    \item $\eta_i \in \mathbb{R}^M$: Gaussian noise with $\eta_i[j] \sim \mathcal{N}(0,\sigma^2) \in \mathbb{R}$
\end{itemize}

% $y_i \in \mathbb{R}^M$ with $M$ as observation dimension.

% Then, $\Pi_z : L^2(\Omega) \to L^2(\tilde{\Omega}), x \mapsto  \int x(\cdot,\cdot,z) dz$ is projection operator from z-axis
% and $Rot : L^2(\Omega) \to L^2(\Omega), Rot(x, \theta_i) = \left((x_1,x_2,x_3) \mapsto x( x_1R^1_{\theta_i}, x_2R^2_{\theta_i}, x_3R^3_{\theta_i})\right)$ is rotation operator modelling the rotation during freezing.
% Further, $\theta_i = [\theta_i^{(1)}, \theta_i^{(2)}, \theta_i^{(3)} ] $ where entries $ \theta_i^{(1)}, \theta_i^{(2)}, \theta_i^{(3)} \in \mathbb{R}$ and 
% $R_{\theta_i} =  R_{e_x} (\theta_i^{(1)}) R_{e_y} (\theta_i^{(2)}) R_{e_z} (\theta_i^{(3)}) = [R^1_{\theta_i}, R^2_{\theta_i}, R^3_{\theta_i}] \in SO(3)$ is the 3D rotation matrix 
% (see \ref{app:3DrotationMatrix} for further details). 
% $\eta_i \sim \mathcal{N}(0,\sigma^2I) \in \mathbb{R}^M$ corresponds to noise of observation.

Equation~\ref{eq:cryoEmSimple} is a simplified version of cryo-EM.
First, point spread function (PSF) of the microscope is not taken into account.
Second, structural variety is ignored, the underlying object $x$ is not the same 
for every observation. 
Precisely, $x$ can be seen as a random signal from an unknown distribution defined over all possible molecules structures.
In this Thesis, only the simplified version in Equation~\ref{eq:cryoEmSimple} is considered.


Morevoer, as $y_i$ is not observable directly, discretization is needed:
\begin{equation}
    \label{eq:cryoEmSimpleDiscrete}
    \begin{aligned}
        y_i &= \left( \Pi_z (\; Rot (\;x; \theta_i)) + \eta_i\right)(\Delta) &, \text{ with } 1 \leq i \leq N \\
        y_i[j,k] &= \Pi_z (\; Rot(\;x; \theta_i))_{j,k} + \eta_i[j,k] &, \text{ with } 1 \leq i \leq N \text{ and } 1 \leq j,k \leq M
    \end{aligned}
\end{equation}

with
\begin{itemize}
    \item $\Delta \subset \tilde{\Omega}^{M}$: sampling grid with dimension $M$
    \item $y[j,k]$, $\eta[j,k]$ and $\Pi_z(\cdot)_{j,k}$ $ \in \mathbb{R}$ with $j,k$ as indices of the sampling grid.
\end{itemize}
% $\Delta \subset \tilde{\Omega}^{M^2}$ is the sampling grid with dimension $M^2$.
% Further, $y[j,k]$, $\eta[j,k]$ and $\Pi_z(\cdot)_{j,k}$ $ \in \mathbb{R}$ with $j,k$ as indices of 
% the sampling grid.


\paragraph{Observation illustration:}
In Figure~\ref{fig:cryo-em-omicron} some cryo-EM observations are illustrated as well as the reconstructed biological sample.

\begin{figure}[H]
    \captionsetup[subfigure]{justification=centering}
    \centering
    \begin{subfigure}[t]{0.2\textwidth}
        \includegraphics[width=\textwidth]{emd_32500.map_xsurface.jpeg}
        \caption{COVID-19 Omicron spike}
    \end{subfigure} \hfill
    \begin{subfigure}[t]{0.2\textwidth}
      \includegraphics[width=\textwidth]{emd_32500.map_xprojection.jpeg}
      \caption{Observation along x-axis}
    \end{subfigure}\hfill
    \begin{subfigure}[t]{0.2\textwidth}
      \includegraphics[width=\textwidth]{emd_32500.map_yprojection.jpeg}
      \caption{Observation along y-axis}
    \end{subfigure}\hfill
    \begin{subfigure}[t]{0.2\textwidth}
        \includegraphics[width=\textwidth]{emd_32500.map_zprojection.jpeg}
        \caption{Observation along z-axis}
      \end{subfigure}
    \caption{Cryo-EM reconstruction and clean projections of COVID-19 Omicron spike \protect\footnote{https://www.ebi.ac.uk/emdb/EMD-32500}}
    \label{fig:cryo-em-omicron}
  \end{figure}


\paragraph{3D cryo-EM reconstruction:}
Cryo-EM reconstruction is defined as estimation of a 3D object from 2D observations.
It can be seen as a 3D problem as the original object $x \in L^2(\Omega)$ to be reconstructed is in 3D.
Based on many observed micrographs the original object $x$ is estimated.
Cryo-EM reconstruction is computational intensive and multiple steps are needed to get from 
observations to the final structure \footnote{For further details \cite{singleParticleCryoEm, cryoEmMath}}.



\section{Abstraction}
As CT and cryo-EM are highly related, the aim in this section is to define
an abstraction model. The big difference is, that CT is in 2D and cryo-EM in 3D.
Mathematically, an extension from 2D to 3D should be theoretically feasible. 
It is considered a numerical question due to more need of resources as computation gets more expensive.
Therefore, an abstract form will be defined.
A similar notation than previously is used, with original object $x \in L^2(\Omega)$.
Further, original object dimension space is parametrized with $D$, consequently $\Omega \subset \mathbb{R}^D$.
Additionally, dimension of observation space is defined as $D-1$, such that 
$\tilde{\Omega} \subset \mathbb{R}^{D-1}$.


\begin{equation}
    \label{eq:abstract-model}
    \begin{aligned}
        y_i &= p_i + \eta_i (\Delta) &, \text{ with } 1 \leq i \leq N \\
        y_i &= \left( A(x, \theta_i) + \eta_i \right) (\Delta) &, \text{ with } 1 \leq i \leq N 
    \end{aligned}
\end{equation}
with
\begin{itemize}
    \item $x \in L^2(\Omega)$: original object
    \item $A: L^2(\Omega) \to \mathbb{R}^M), x \mapsto A(x; \theta_i)$: a non-linear operator 
    \item $\theta_i \in SO(P)$: projection angle(s) vector, with $P$ projection dimension
    \item $\eta \sim \mathcal{N}(0, \sigma^2 I) \in \tilde{\Omega}^M$: i.i.d Gaussian noise
    \item $\Delta \subset \tilde{\Omega}^{M}$: term for discretization
\end{itemize}

\paragraph{Reconstruction:}
\textbf{TODO: 
equation 3.7, Delta is Omega tilde power M not power D.
Also, as before, theta is in SO(3) (in equation 3.3 also theta should be in [0,2pi], maybe with finding a consistent definition)}


Further, an abstract form of the reconstruction operator is defined as:

\begin{equation}
    \textit{Recon} : L^2(\tilde{\Omega}) \to L^2(\Omega), y \mapsto Recon(y; \theta)
\end{equation}

with
\begin{itemize}
    \item $\theta_i \in \mathbb{R}^P$: projection angle(s) vector, with $P$ projection dimension
\end{itemize}

\paragraph{Classical tomography:}
Classical tomography parameters are defined with $D=2$, $P=1$.
Further, $A(\cdot)$ is the Radon Transform (see Equation~\ref{eq:2Dreconstruction}).
Reconstruction operator can be defined as FBP (with or without U-Net).

\paragraph{Cryo-EM:}
Cryo-EM parameters are defined with $D=3$ and $P=3$ as $\theta_i$ not only corresponds to
a projection angle vector but also some rotation.
Further, $A(\cdot)$ can be defined as $\Pi_z \left(\; \textit{Rot}(\;x; \theta) \right)$ 
where \textit{Rot} is the 3D rotation and $\Pi_z$ the tomographic projection.


\paragraph{High noise regime:}
Cryo-EM observations are highly noisy, which makes reconstruction challenging. 
There are different ways to reduce noise from observations, most of them are related to averaging. 
Averaging needs to consider similar observations and ignore diverse ones. 
In the defined abstract model, averaging over paired observations from $\theta$ should be a good averaging model.

One idea would be to measure distances between observations.
Another way is to find a low-dimensional embedding which maps observation $y$ to $\theta$.
When talking from low-dimensional embeddings, there is no way around Graph Learning, which will be introduced
in the following Chapter~\ref{sec:manifold_and_graphs} \textit{\nameref{sec:manifold_and_graphs}}.
