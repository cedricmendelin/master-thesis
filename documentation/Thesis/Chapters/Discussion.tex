\chapter{Conclusion and Future Work}
\label{sec:Conclusion}

In this final chapter, the conclusion and future work is presented.

\section{Conclusion}

In this Thesis, the molecular imaging methods CT and Cryo-EM
within the high noise domain, have been the problems to approach.

During the practical part, 2D and 3D tomography have been implemented 
in python to familiarize with the domain.
Further, vector diffusion map was implemented in 3D to do some comparison
and code was integrated with Aspire\footnote{\url{http://spr.math.princeton.edu/}}.
Finally, GAT was implemented and further optimized with convolution and U-Net.
Finally, GAT-Denoiser is derived.

During evaluation on the LoDoPab-CT dataset, baseline algorithm BM3D was implemented.
Overall, GAT-Denoiser shows great results and improves reconstruction SNR 
by 379.9\%, 126.0\% 57.7\%, 27.6\% for \snry -15 dB, -10 dB, -5 dB and 0 dB.
GAT-Denoiser is able to capture important observation information even in the high-noise domain.
Moreover, the individual components contribute to the success of GAT-Denoiser.
With the small scale experiments, 
it was shown that convolution, GAT and the end-to-end learning approach are all valuable individually
and are therefore well-chosen.
Additionally, the large scale experiments found the best GAT-Denoiser model.
GAT-Denoiser combined with U-Net and joint neural network training can upgrade CT reconstruction 
in the high-noise domain to a new level. 

\section{Future Work}
Cryo-EM is an open research area of great interest.
The 3D problem is not easy to solve and there are many steps in the Cryo-EM reconstruction pipeline 
to be further improved. With GAT-Denoiser, a neural network architecture is introduced, 
towards an algorithm to work for Cryo-EM

\paragraph{Improve GAT-Denoiser:}
The current GAT-Denoiser is not perfect. 
In the architecture, a single convolutional layer proceeds a single GAT layer.
In some scenarios, it could be beneficial, to apply multiple convolutional layers before GAT and 
on could resolve the current architecture to allow multiple convolutional layer before GAT.
Further, on could think of integrating convolution in GAT directly or adjust the architecture
in other manners.


\paragraph{GAT-Denoiser in 3D:}
So far, GAT-Denoiser was evaluated on a CT dataset in 2D.
Therefore, the next step would be to increase the dimension and 
derive GAT-Denoiser to work in 3D. This should be feasible without too much of an effort,
as the single components of GAT-Denoiser are able to work in 3D as well. 
However, more resources will be needed for computation.

\paragraph{Drop known Angle Assumption:}
So far, GAT-Denoiser works with the assumption that angles are fixed.
In a future work, this assumption could be dropped. GL embedding
is a way to approximate angles for CT and Cryo-EM observations, but the quality in the high noise regime is rather low.
To extend GAT-Denoiser to work with unknown angles, it needs to be combined with an additional component, 
which can successfully approximate angles in the high noise regime.

\paragraph{Further Cryo-EM Challenges:}
In this Thesis, only single particle Cryo-EM is considered.
When GAT-Denoiser is available in 3D, one could think of an approach
where from a given observation, the underlying sample is not 
structural equivalent and there are some slight differences. 
One needs to approximate the distribution of the underlying samples.
This could be established first for known observation angles and further brought to unknown observation angles.

