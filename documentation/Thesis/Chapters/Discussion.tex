\chapter{Conclusion and Future Work}
\label{sec:Conclusion}

In this final chapter, conclusion and future work is presented.

\section{Conclusion}

In this Thesis, the molecular imaging methods CT and cryo-EM
within the high noise domain, have been the problems to approach.

During the practical part, 2D and 3D tomography have been implemented 
in python to familiarize with the domain.
Further, vector diffusion map was implemented in 3D to so some comparison
and code was integrated with Aspire\footnote{\url{http://spr.math.princeton.edu/}}.
Finally, GAT was implemented and further optimized with convolution and U-Net.
Therefore, GAT-Denoiser was introduced.

During evaluation on the LoDoPab-CT dataset, baseline algorithm BM3D was implemented.
Overall, GAT-Denoiser shows great results and improves reconstruction SNR 
by 379.9\%, 126.0\% 57.7\%, 27.6\% for \snry -15 dB, -10 dB, -5 dB and 0 dB.
GAT is able to capture important observation information even in the high-noise domain.
Further, GAT-Denoiser combined with U-Net and joint neural network training
can boost CT reconstruction. 




\begin{itemize}
  \item GAT for denoising works with Manifold Assumption
  \item U-Net boosts performance
  \item U-Net could even be trained better (not SNR interval, longer)
\end{itemize}

\section{Future Work}

Cryo-EM is an open research area of great interest.
The 3D problem is not easy to solve and there are many steps to further improve.

\paragraph{GAT-Denoiser in 3D:}
So far, GAT-Denoiser was evaluated on a CT dataset in 2D.
Therefore, the next step would be to increase the dimension and 
derive GAT-Denoiser to work in 3D. 

\paragraph{Drop known angle assumption:}
So far, GAT-Denoiser works with the assumption that angles are fixed.
In a future work, this assumption could be dropped. GL embedding
is a way to approximate angles for CT and cryo-EM observations.
The challenge is, that the embedding gets
 useless when dealing with too much noise.

\paragraph{Further cryo-EM challenges:}
In this Thesis, only single particle cryo-EM is considered.
When GAT-Denoiser is available in 3D, one could also think of to approach
the problem where from a given observation, underlying sample is not 
structural equivalent and there are some slight differences. 

