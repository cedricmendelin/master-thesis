% !TEX root = ../Thesis.tex
\addtocontents{toc}{\protect\setcounter{tocdepth}{0}}
\chapter{Appendix}

\subsection{signal-to-noise-ratio (SNR)}

\textbf{TODO:}

\subsection{3D rotation matrix}
\label{app:3DrotationMatrix}
A rotation matrix is a transformation matrix used to perform rotations.
In 3D case, matrix for rotating one single axis can be described as:
\begin{equation}
    R_{e_x} (\theta)
    \begin{bmatrix}
        1 & 0 & 0\\
        0 & \cos \theta & - \sin \theta \\
        0 & \sin \theta & cos \theta \\
    \end{bmatrix}
\end{equation}

\begin{equation}
    R_{e_y} (\theta)
    \begin{bmatrix}
        \cos \theta & 0 & \sin \theta\\
        0 & 1 & 0 \\
        - \sin \theta & 0 & cos \theta \\
    \end{bmatrix}
\end{equation}

\begin{equation}
    R_{e_z} (\theta)
    \begin{bmatrix}
        \cos \theta & - \sin \theta\\
        \sin \theta & \cos \theta & 0 \\
        0 & 0 & 1 \\
    \end{bmatrix}
\end{equation}

where $e_x, e_y, e_z$ corresponds to the axis unit-vector (for x: $(1,0,0)$, etc.) and $\theta \in \mathbb{R}$.
To combine the single axis rotations, matrices can be multiplied with each other:

\begin{equation}
    \label{eq:3d-rotation}
    R (\theta) = R_{e_x} (\theta) R_{e_y} (\theta) R_{e_z} (\theta)
\end{equation}

In Equation~\ref{eq:3d-rotation}, angle $\theta$ is the same for all axis, which does not have to be.


\subsection{Power Iterations}
\label{sec:powerIterations}

Power iteration, also called power method, is an iterative method
that approximates the largest eigenvalue of a diagonalizable matrix $A$.

The algorithm starts with a random vector $b_0$ or an approximation of the dominant eigenvector.

\begin{equation}
    \label{eq:powerIterations}
    b_{k+1} = \frac{Ab_k}{||Ab_k||}
\end{equation}

It will not necessarily converges. The algorithm will converge if $A$ has an eigenvalue strictly grater than its other eigenvalues
and initial vector $b_0$ is not orthogonal to the eigenvector associated with the largest eigenvalue.

\section{Small Experiment: additional results}

\textbf{TODO: Add Loss? cut numbers}

\begin{table}[H]
    \centering
    \begin{tabular}{l|c|c|c|c}
      \toprule
      \textbf{Algorithm} & \snrh{ 0} & \snrh{ -5} & \snrh{ -10} & \snrh{ -15} \\
                         & \textbf{SNR} & \textbf{SNR} & \textbf{SNR}  & \textbf{SNR} \\ 
      \midrule
      FBP&4.80&-0.017&-4.96&-9.94 \\ \hline
      U-Net &9.03&8.17&6.06&3.10 \\ \hline
      BM3D Reco& 10.15&6.81&2.88&-1.42 \\ \hline
      BM3D Sino&9.30&6.14&2.93&-0.50 \\ \hline
      GAT&8.61&7.32&5.86&4.50 \\ \hline
      Conv + GAT&9.08&8.27&7.51&5.88 \\ \hline
      Gat + Unet&8.72&7.72&7.01&5.48 \\ \hline
      Gat + Unet*&12.86&11.52&10.84&8.94 \\ \hline
      Conv + Gat + Unet&8.75&7.34&6.14&5.15 \\ \hline
      Conv + Gat + Unet*&12.10&10.33&8.58&7.06 \\
      \midrule
    \end{tabular}
  
    \caption{Small Experiment: overall GAT-Denoiser components vs. Baseline}
    \label{tab:baseline-large}
  \end{table}
  
  
  

\subsection{GAT-Denoiser components}

\section{Large Experiment: additional results}


\addtocontents{toc}{\protect\setcounter{tocdepth}{2}}