% !TEX root = ../Thesis.tex
\chapter{Abstract}


Cryo-electron microscopy (Cryo-EM) is a modern molecular imaging modality which has recently enabled atomic resolution reconstructions of large biological molecules. 
Unlike in diffraction crystallography which requires challenging crystallization of large molecules, 
in Cryo-EM we freeze molecules in a thin layer of vitreous ice without any need to form crystals. 
This led to Cryo-EM becoming the leading technology to determine 3D protein structures and its inventors were awarded the 2017 Nobel Prize in Chemistry.

Cryo-EM is a transmission modality: the electron beam passes through the sample and becomes attenuated where the electron density of the molecule is large. 
Since the sample contains many identical (to the first approximation) copies of the same protein and various random orientations, 
the collected projection data lends itself to tomographic inversion, with the caveat that the orientations are a priori unknown. 
Further, in order to avoid damaging the sample the electron dose must be kept low, which leads to very noisy projections. 
Taken together, unknown orientations and very large noise require much more sophisticated reconstruction algorithms than in standard tomography.

In this thesis we tackle the projection denoising problem. 
We propose GAT-Denoiser, a graph neural network which combines graph attention with convolutional updates and physics-based end-to-end learning. 
In this proof of concept work we assume that the orientations are known, and we work in 2D, focusing tightly on the problem of noise.
We demonstrate through extensive experiments that GAT-Denoiser brings about substantial improvements over strong baselines, especially in the very low SNR regime.
Moreover, we show that all components of the GAT-Denoiser network contribute to its strong performance. 
In particular, the convolutional filters, the graph attention, and end-to-end learning are all essential and the absence of any of them deteriorates performance.
Finally, we show that jointly training the GAT-Denoiser with a reconstruction U-Net brings about further improvements.
This work may be a foundation of a full Cryo-EM reconstruction pipeline in which the graph connectivity is inferred from noisy projections rather than given a priori.
