% !TEX root = ../Thesis.tex
\chapter{Abstract}


In molecular imaging methods Cryo-Electron Microscopy (Cryo-EM) and Computed Tomography (CT)
observations are collected from a biological sample. 
Cryo-EM enables the view of molecules in near-atomic resolution and, therefore, has received
a lot of attention in recent years.
The Application of Cryo-EM is a major motivation for this Thesis. 
Observations are noisy due to the imaging process and reconstruction from noisy observations is desired.
Reconstruction is challenging due to dealing with enormous noise and unknown observation angles.
Moreover, Cryo-EM can be seen as a problem in 3D, as the biological sample is imaged in 3D.
CT is similar to Cryo-EM, but reconstruction is slightly simpler, since it is potentially in 2D and 
observation angles are known.
That's why it is well suited as a first step towards a Cryo-EM algorithm.

I introduce \textit{GAT-Denoiser}, a Graph Neural Network which combines Graph Attention Network (GAT) with convolution 
and an end-to-end learning approach. 
It enables denoising of observations and improves reconstruction quality with the assumption of known projection angles.
During training, an end-to-end approach is applied, that compares reconstruction quality.
GAT-Denoiser has been evaluated on the LoDoPaB-CT dataset.
It could outperform baseline algorithm BM3D measured by the reconstruction
signal-to-noise-ratio (SNR). An improvement by 379.9\%, 126.0\% 57.7\%, 27.6\% for SNR -15 dB, -10 dB, -5 dB and 0 dB, respectively
could be reached.
Moreover, the individual components of GAT-Denoiser contribute to the success of GAT-Denoiser.
During a series of experiments, it was shown that convolution, GAT and end-to-end learning are all valuable individually.
Further, GAT-Denoiser combined with U-Net and joint neural network training can upgrade CT reconstruction in the high-noise domain to a new level. 

