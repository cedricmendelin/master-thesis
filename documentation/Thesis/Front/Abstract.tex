% !TEX root = ../Thesis.tex
\chapter{Abstract}


Cryo-Electron Microscopy (cryo-EM) and Computer Tomography (CT) 
are two molecular imaging methods.
In both scenarios, observations are collected from a biological sample.
During reconstruction, the aim is, to approximate the underlying biological sample from observations.
For CT, the reconstruction can be in 2D or 3D, for cryo-EM it is in 3D.
Cryo-EM enables the view of molecules in near-atomic resolution and therefore got a lot of attention in recent years.
Due to ground-breaking improvements regarding hardware and data processing, the field of research
has highly improved.
Reconstruction is challenging due to the present of enormous noise and unknown projection angles in the observations.
CT is a similar to cryo-EM, but reconstruction is slightly easier
as the problem can be in 2D and observation angles are known. 
Therefore, CT is well suited to implement an algorithm toward an extended version working for cryo-EM as well.

I introduce GAT-Denoiser, a Graph Neural Network, which combines Graph Attention Network with convolution 
and an end-to-end learning approach. 
It enables to denoise observations and boosts reconstruction quality with the assumption of known projection angles.
During training an end-to-end approach is used where reconstruction quality is compared and not only
denoised observation quality.
GAT-Denoiser have been evaluated on the LoDoPaB-CT~\cite{lodopab-dataset} dataset.
It could outperform baseline algorithm BM3D measure by the reconstruction
signal-to-noise-ratio (SNR). An improvement by 379.9\%, 126.0\% 57.7\%, 27.6\% for \snry -15 dB, -10 dB, -5 dB and 0 dB respectively
could be fulfilled. 

