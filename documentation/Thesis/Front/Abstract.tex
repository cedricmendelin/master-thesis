% !TEX root = ../Thesis.tex
\chapter{Abstract}


Cryo-Electron Microscopy (cryo-EM) and Computer Tomography (CT) 
are two molecular imaging methods.
In both cases, observations are collected from a biological sample.
During reconstruction, the aim is, to approximate
 the underlying biological sample.
 Due to the present of enormous noise and unknown projection angles 
in the observations, reconstruction is challenging.

GAT-Denoiser is a new approach to combine Graph Attention Network 
with convolution. It enables to denoise observations and boosts
reconstruction quality for the assumption of known projection angles.
During training and end-to-end approach 
is used where reconstruction quality is compare and not only
denoised observation quality.
GAT-Denoiser have been evaluated on the LoDoPaB-CT~\cite{lodopab-dataset} dataset,
where it could outperform baseline algorithm BM3D 
reconstruction SNR by 379.9\%, 126.0\% 57.7\%, 27.6\% for \snry -15 dB, -10 dB, -5 dB and 0 dB. 
