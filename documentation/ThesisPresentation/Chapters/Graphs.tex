
\section{Graphs \& Manifolds}

\begin{frame}{Graph - Definitions}
  \pause
  \begin{block}{Graph Definition}
    A graph is defined as $G = \langle V,E \rangle$, where $V$ is a set of 
    nodes and $E$ is a set of edges. 
  \end{block}


  \begin{columns}
    \column{.55\textwidth}
    \only<2->{
      \begin{block}{Nodes}
        $(n_1, n_2, \dots) \in \mathbb{R}^F$, with $F$ as node feature dimensions
      \end{block}
    
      \begin{block}{Edges}
        Edges are defined as a set of tuples $(i, j)$, where $i$ and $j$ determine 
        the index of the nodes.
      \end{block}
    }

    
    \column{.45\textwidth}
    \only<3->{
      \begin{figure}
        \centering
        \includegraphics[width=\textwidth]{graph_undirected.drawio.png}
        \caption{Sample graph}        
      \end{figure}
    }
    
  \end{columns}
\end{frame}

\begin{frame}{Graph - Definitions - Adjacency Matrix}
  \pause
  \begin{columns}
  \column{.55\textwidth}

  \begin{block}{Adjacency Matrix}
    The binary adjacency matrix of graph $G = \langle V, E \rangle$ is defined as:
  \begin{equation}
      \label{eg:AdjacencyMatrix}
      A_{ij} =    
      \begin{cases}
          1  & \text{if } (i, j) \in E \\
          0, & \text{otherwise}
      \end{cases}
  \end{equation}
  \end{block}
  
  \column{.45\textwidth}
  \begin{figure}
    \centering
    \includegraphics[width=0.9\textwidth]{graph_A.drawio.png}
    \caption{Sample graph}        
  \end{figure}

\end{columns}

\end{frame}

\begin{frame}{Graph - Definitions - Degree}
  \pause
  \begin{columns}
    \column{.55\textwidth}
  
    \begin{block}{Degree of a node}
      The $degree$ of a node is defined as the number of (incoming) edges.
    \end{block}

    \only<3->{
      \begin{block}{Degree Matrix of Graph $G$}
        Is a diagonal matrix with degree of nodes as entries.
        $$D_{ii} = degree(n_i)$$
      \end{block}
    }

    \column{.45\textwidth}
    \only<2->{
      \begin{figure}
        \centering
        \includegraphics[width=0.9\textwidth]{graph_degree.drawio.png}
        \caption{Sample graph}        
      \end{figure}
    }
    
  \end{columns}
  \end{frame}

\begin{frame}{Graph for Molecular Imaging Observation}
  How to construct a graph for molecular imaging?
  \pause
  \begin{columns}
    \column{0.45\textwidth}
    \begin{itemize}
      \item Nodes: Single observation $y_i$
      \item<3-> Edges: Use k-nearest neighbours (k-NN) to construct a graph
      \begin{itemize}
        \item<4-> Define similarity measure for nodes: $\ell2$-norm
        % \begin{itemize}
        %   \item<5-> CT:  $\norm{y_i - y_j}^2_2$ 
        %   \item<5-> Cryo-EM: $min_{g \in SO(2)}\norm{g \;y_i - y_j}^2_2$
        % \end{itemize}
      \end{itemize}
    \end{itemize}

    \column{0.55\textwidth}
    \begin{figure}
      \centering

      \only<2-3>{
        \includegraphics[width=0.8\textwidth]{cryo-EM_graph_nodes.drawio.png}
      }
      \only<4>{
        \includegraphics[width=0.8\textwidth]{cryo-EM_graph_nodes_k.drawio.png}
      }

      \only<5>{
        \includegraphics[width=0.7\textwidth]{cryo-EM_graph_node_distance.drawio.png}
      }

      \only<6>{
        \includegraphics[width=0.8\textwidth]{cryo-EM_graph.neighbours_n2.drawio.png}
      }

      \only<7>{
        \includegraphics[width=0.8\textwidth]{cryo-EM_graph.drawio.png}
      }
      \caption{Sample graph for cryo-EM observation}        
    \end{figure}
  \end{columns}
\end{frame}

\begin{frame}{Graph for Molecular Imaging Observation - Noise}
  What happens with our noisy observations?
  \pause
  \begin{columns}
    \column{0.45\textwidth}
    \begin{itemize}
      \item With noise, graph will capture neighborhood inaccurately.
    \end{itemize}

    \column{0.55\textwidth}
    \begin{figure}
      \centering
      \includegraphics[width=0.8\textwidth]{cryo-EM_graph_noisy.drawio.png}
      \caption{Sample graph for noisy cryo-EM observation}        
    \end{figure}
  \end{columns}
\end{frame}


\begin{frame}{Graph Laplacian (GL)}
  What can we use this graph for?
  \pause
  \begin{itemize}
  \item \cite{LaplaceRandomProjections} used it to approximate angles for CT:
\end{itemize}

\pause

\begin{block}{Low-dimensional Embedding}
  \begin{enumerate}
    \item Construct a k-NN graph from observations.
    \item Calculate $L = D - A $
    \item Get 2nd and 3rd smallest eigenvalue with corresponding eigenvectors.
  \end{enumerate}
\end{block}

\end{frame}


\begin{frame}{Low-dimensional Embedding for Computed Tomography}
  \begin{figure}
    \centering
    \begin{subfigure}[t]{0.3\textwidth}
        \includegraphics[width=\textwidth]{phantom_sino.png}
        \caption{Clean CT observation}
    \end{subfigure} \hfill
    \pause
    \begin{subfigure}[t]{0.3\textwidth}
        \includegraphics[width=\textwidth]{graph.drawio.png}
        \caption{Building k-NN graph with $ k = 2$ }
    \end{subfigure}\hfill
    \pause
    \begin{subfigure}[t]{0.3\textwidth}
      \includegraphics[width=\textwidth]{phaton_clean_manifold.png}
      \caption{$2_{nd}$ and $3_{rd}$ smallest eigenvectors of  $L =  D - A$}
  \end{subfigure}
\end{figure}
  
\end{frame}


\begin{frame}{Computed Tomography with unknown angles}
  \begin{itemize}
    \item GL-Embedding estimates observation angles
  \end{itemize}

  \pause
  \begin{figure}
    \centering
    \only<2-4>{
      \begin{subfigure}[t]{0.3\textwidth}
        \includegraphics[width=.8\textwidth]{fbp_phantom_clean.png}
        \caption{Reconstruction known angles}
      \end{subfigure} \hfill
      \pause
      \begin{subfigure}[t]{0.3\textwidth}
          \includegraphics[width=0.8\textwidth]{phaton_clean_manifold.png}
          \caption{GL-Embedding from $k=2$}
      \end{subfigure}\hfill
      \pause
      \begin{subfigure}[t]{0.3\textwidth}
        \includegraphics[width=0.8\textwidth]{fbp_phantom_clean_unknown_angles.png}
        \caption{Reconstruction unknown angles}
    \end{subfigure}
    }
    
    \only<5>{\alert{What happens in the noisy case?}}

    \only<6>{
      \begin{subfigure}[t]{0.3\textwidth}
        \includegraphics[width=.8\textwidth]{fbp_phantom_snr_10.png}
        \caption{Reconstruction known angles $SNR_y$ : 10 dB}
      \end{subfigure} \hfill
      \begin{subfigure}[t]{0.3\textwidth}
          \includegraphics[width=.8\textwidth]{phaton_noisy_manifold_k6_snr10.png}
          \caption{GL-Embedding from $k=6$ and $SNR_y$ : 10 dB }
      \end{subfigure}\hfill
      \begin{subfigure}[t]{0.3\textwidth}
        \includegraphics[width=.8\textwidth]{fbp_phantom_snr_10_unknown_angels.png}
        \caption{Reconstruction unknown angles $SNR_y$ : 10 dB}
      \end{subfigure}
    }

    \only<7>{
      \begin{subfigure}[t]{0.3\textwidth}
        \includegraphics[width=.8\textwidth]{fbp_phantom_snr_0.png}
        \caption{Reconstruction known angles $SNR_y$ : 0 dB}
      \end{subfigure} \hfill
      \begin{subfigure}[t]{0.3\textwidth}
          \includegraphics[width=.8\textwidth]{phaton_noisy_manifold_k6_snr0.png}
          \caption{GL-Embedding from $k=6$ and $SNR_y$ : 0 dB }
      \end{subfigure}\hfill
      \begin{subfigure}[t]{0.3\textwidth}
        \includegraphics[width=.8\textwidth]{fbp_phantom_snr_0_unknown_angels.png}
        \caption{Reconstruction unknown angles $SNR_y$ : 0 dB}
      \end{subfigure}
    }
  \end{figure}

\end{frame}



% \begin{frame}[c]{Graph Denoising}
%   \pause
%     \textit{Graph denoising} is the task, to approximate a graph $G$ from a given noisy version graph $G_0$:

%     \begin{definition}[Graph Denoising]
%         $$GD: G_0 \mapsto \tilde{G} \approx G$$
%     \end{definition}
%     where $G_0$, $\tilde{G}$, $G$ denotes noisy, denoised and original graph respectively.
% \end{frame}

\begin{frame}{Observation Denoising}
  The fewer noise is available in the observation, the better reconstruction is possible.
  
  \pause
  \begin{itemize}
    \item Use existing denoising algorithms
    \begin{itemize}
      \item Block-matching and 3D filtering (BM3D) \cite{bm3d}
      \item Non-local means \cite{noneLocalMean}
      \item No graph as data structure
      \item But, both exploit neighborhood during averaging
    \end{itemize}
    \item<3-> \alert<3->{Shows potential for graph as a data structure}
  \end{itemize}

  \begin{tcolorbox}[colback=red!5!white,hide=<1-3>, alert=<4>, colframe=red!75!black]
    Exploit graph as a data structure and the GL-Embedding
\end{tcolorbox}

  
\end{frame}
