
\section{Molecular Imaging Methods}	% You can also have slides prior to the first section or work entirely without sections.

\begin{frame}[c]{Cryo-Electron Microscopy (Cryo-EM)}
    \begin{itemize}
        \item Enables observation of molecules in near atomic resolution.
        \item Major motivation for Thesis.
        \item During freezing, molecules rotate randomly.
        \item Frozen molecules are fragile, electron microscope needs to work with low power.
        \item Observations can be reconstructed to 3D model.
    \end{itemize}

    \begin{tcolorbox}[colback=red!5!white,hide=<-1>, alert=<2>, colframe=red!75!black]
        Only single particle cryo-EM is considered.
    \end{tcolorbox}

\end{frame}

\begin{frame}[c]{Cryo-EM}
    \begin{figure}
        \captionsetup[subfigure]{justification=centering}
        \centering
        \hfill
        \begin{subfigure}[t]{0.35\textwidth}
            \vskip 0pt
            \includegraphics[width=\textwidth]{CryoObservation.drawio.png}
        \end{subfigure}\hfill
        \pause
        \begin{subfigure}[t]{0.4\textwidth}
            \vskip 0pt
            \includegraphics[width=\textwidth]{Cryo-EM_reconstruction.drawio.png}
        \end{subfigure}\hfill
        \caption{Cryo-EM overview}
    \end{figure}

\end{frame}

\begin{frame}[c]{Cryo-EM challenges}
    \begin{columns}[c]
        \column{.45\textwidth}
        
        \begin{itemize}
            \item High-noise level 
            \item Unknown rotation during freezing
            \item (Structural variety of observations)
        \end{itemize}

        % \begin{tcolorbox}[colback=red!5!white,hide=<-1>, alert=<2>,colframe=red!75!black]
        %     Master Thesis domain of interest is to high-noise regime (cryo-EM).
        %     Goal is to introduce a denoise method for cryo-EM 2D projections.
        % \end{tcolorbox}

        \column{.55\textwidth}
        \begin{figure}
            \centering
            \begin{subfigure}[t]{0.5\textwidth}
                \includegraphics[width=\textwidth]{bunny_872.png}
                \caption{Clean micrograph}
            \end{subfigure}\hfill                
            \begin{subfigure}[t]{0.5\textwidth}
                \includegraphics[width=\textwidth]{bunny_872_noisy.png}
                \caption{Noisy micrograph}
            \end{subfigure}\hfill                
        \end{figure}

    \end{columns}

    %\footnotetext[1]{https://www.ebi.ac.uk/emdb/EMD-32177}
\end{frame}

%\begin{frame}
%    \begin{definition}[Cryo-EM observation]
%        $$ y_i[j,k] = \Pi_z (\; Rot(\;x; \theta_i))_{j,k} + \eta_i[j,k], \text{ with } 1 \leq i \leq N \text{ and } 1 \leq j,k \leq M,$$
%    \end{definition}
%    \begin{itemize}
%        \item $y_i[] \in \tilde{\Omega}, x \in L^2(\Omega)$ with $\Omega \subset \mathbb{R}^3 $ and $\tilde{\Omega} \subset \mathbb{R}^2 $
%        \item $M$ observation dimension
%        \item $\Pi_z : L^2(\Omega) \to L^2(\tilde{\Omega})$ projection operator
%        \item $Rot : L^2(\Omega) \to L^2(\Omega),$ is rotation operator
%        \item $Rot(x, \theta_i) = \left((x_1,x_2,x_3) \mapsto x( x_1R^1_{\theta_i}, x_2R^2_{\theta_i}, x_3R^3_{\theta_i})\right)$
%        \begin{itemize}
%            \item $\theta_i = [\theta_i^{(1)}, \theta_i^{(2)}, \theta_i^{(3)} ] $, with $\theta_i^{(1)}, \theta_i^{(2)}, \theta_i^{(3)} \in \mathbb{R}$
%            \item $R_{\theta_i} =  [R^1_{\theta_i}, R^2_{\theta_i}, R^3_{\theta_i}] \in SO(3)$ is the 3D rotation matrix 
%        \end{itemize}
%    \end{itemize}
%\end{frame}


\begin{frame}[c]{Computed Tomography (CT)}
    \begin{columns}[c]
        \column{.55\textwidth}
            \begin{itemize}
                \item Related to cryo-EM
                \item Can be seen as a simpler version in 2D
                \item Good to start with towards a cryo-EM algorithm
            \end{itemize}
        
        \column{.45\textwidth}
        \begin{figure}
            \centering
            \begin{subfigure}[t]{0.45\textwidth}
                \includegraphics[width=\textwidth]{ct_im_2}
                \caption{Biological sampel}
            \end{subfigure}\hfill                
            \begin{subfigure}[t]{0.51\textwidth}
                \includegraphics[width=\textwidth]{phantom_sino}
                \caption{clean Sinogram}
            \end{subfigure}\hfill          
        \end{figure}

        
    \end{columns}

\end{frame}

\begin{frame}{Observation}

    \begin{block}{Observation}
        \begin{equation}
            \begin{aligned}
                y &= p + \eta \\
                y_i &= \left( A(x, \theta_i) \right) \\
                y_i[j] &= p_i[j] + \eta_i[j] & \text{ with } 1 \leq i \leq N, 1 \leq j \leq M  \\
            \end{aligned}
        \end{equation}

    \end{block}

    \begin{columns}[T]
    \column{.5\textwidth}

    \begin{itemize}
        \item $y$: noisy observation
        \item $p$: noiseless observation
        \item $\eta$: noise, assumed  $\eta_i \sim \mathcal{N}(0,\sigma^2)$
        \item $\theta_i$: observation angle
    \end{itemize}
        
    \column{.5\textwidth}

    \begin{itemize}

        \item $N$: number of observations
        \item $M$: observation dimension
        \item $A: L^2(\Omega) \to \mathbb{R}^M, x \mapsto A(x; \theta_i)$: \\a non-linear operator 
    \end{itemize}

    \end{columns}
\end{frame}

\begin{frame}{Observation - Computed Tomography }

    \begin{figure}
    \centering
    \begin{subfigure}{0.4\textwidth}
        \includegraphics[width=\textwidth]{phantom.png}
        \caption{Biological Sample}
    \end{subfigure}
    \begin{subfigure}{0.4\textwidth}
        \includegraphics[width=\textwidth]{sino_yi.drawio.png}
        \caption{CT Observation - sinogram}
    \end{subfigure}
\end{figure}

\end{frame}


\begin{frame}{Observation - Cryo-EM }

    \begin{figure}
        \centering
        \begin{subfigure}[t]{0.4\textwidth}
            \includegraphics[width=\textwidth]{bunny.PNG}
            \caption{Biological Sample}
        \end{subfigure}
        \begin{subfigure}[t]{0.4\textwidth}
            \includegraphics[width=\textwidth]{micrograph_yi.drawio.png}
            \caption{Cryo-EM Observation - micrographs}
        \end{subfigure}
    \end{figure}

\end{frame}



\begin{frame}{Reconstruction}

    \begin{columns}
        \column{.45\textwidth}
        \begin{block}{Reconstruction}
            \begin{equation}
                \begin{aligned}
                    \textit{Recon} : & \mathbb{R}^{M \times N} \to \mathbb{R}^{M \times M} \\
                    & y \mapsto Recon(y; \theta)
                \end{aligned}
            \end{equation}
        \end{block}

        \column{.55\textwidth}
            
        \begin{figure}
            \centering
            \begin{subfigure}[t]{0.45\textwidth}
                \includegraphics[width=\textwidth]{fbp_phantom_clean.png}
                \caption{Reconstruction clean: \\
                    $Recon(p, \theta) \approx x$}
            \end{subfigure}
            \begin{subfigure}[t]{0.45\textwidth}
                \includegraphics[width=\textwidth]{fbp_phantom_snr_0.png}
                \caption{Reconstruction noisy: \\
                    $Recon(p, \theta) \not\approx x$}
            \end{subfigure}
        \end{figure}

    \end{columns}

\end{frame}

\begin{frame}{Problem and Goal}
    
    \begin{block}{Problem}
        $p$ not observable directly only $y$ is observable.
    \end{block}

    \pause
    \begin{block}{Goal}
        $$ denoiser:   (p_i + \eta) \mapsto y_i^* \approx y_i $$
        \pause
        $$ \textit{Recon} \left( denoiser(y; \theta) \right) \approx x $$
    \end{block}

\end{frame}


