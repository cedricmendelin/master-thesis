\chapter{Project Plan}
\label{sec:projectPlan}
In the last chapter of the Thesis Preparation report, the project plan will be introduced as well as a broad overview of different 
work packages. Further, the project timeline can be seen as a Gantt chart.
Probably, there are some parts which will not work out as expected and 
adjustments are needed throughout the Thesis, the project plan can be seen as a rough guideline.

\section{Work packages}


\paragraph{Implement algorithm for 2D case:}
The first step will be, to familiarize with the problem and implement
the algorithm for 2D. 

\paragraph{Evaluate 2D case on toy dataset and implement baselines:}
As a second step, the implemented 2D algorithm will be tested on a toy dataset,
where noise is added to the images by hand. As the aim is to work with highly noisy images,
the noise level can be selected and increased when working with toy datasets. 
The evaluation in 2D is crucial and needs to in a satisfying matter. 
It does not make sense to continue with 3D implement, when the simply 2D case is not handled well enough.
Therefore, if the evaluation results are not satisfying, the algorithm needs to be iteratively adjusted, 
such that the evaluation will be in a good enough quality.


\paragraph{Implement algorithm for 3D case:}
After successfully evaluating the algorithm in 2D, the aim is to extend the algorithm to work in 3D as well.

\paragraph{Evaluate 3D case on toy dataset and adjust baselines:}
Again, the implementation will be evaluated on a toy dataset, where noise can be adjusted by hand.


\paragraph{Nice to have: Evaluate of real dataset}
If time allows and the 2D and 3D implementation are evaluated successfully on toy datasets, 
real data can be used for further evaluation. This step will only be done, if time allows.


\paragraph{Evaluate related work:}
As cryo-EM reconstruction is a hot research topic, related work can not only
be considered during the start of the Thesis and needs to be evaluated throughout the Thesis.

\paragraph{Writing Thesis:}
Document implementation and evaluation result.

\begin{landscape}

    \section{Gantt chart}

    \newganttchartelement{bluebar}{
    bluebar/.style={
        inner sep=0pt,
        draw=purple!44!black,
        very thick,
        top color=white,
        bottom color=blue!80
    },
    bluebar label font=\slshape,
    bluebar left shift=.1,
    bluebar right shift=-.1
}

\newganttchartelement{greenbar}{
    greenbar/.style={
        inner sep=0pt,
        draw=green!50!black,
        very thick,
        top color=white,
        bottom color=green!80
    },
    greenbar label font=\slshape,
    greenbar left shift=.1,
    greenbar right shift=-.1
}

\ganttset{%
calendar week text={W\currentweek}%
}

    
    \begin{adjustwidth}{-70pt}{}
        \begin{ganttchart}[expand chart=1.8\textwidth,
            hgrid style/.style={black, dashed},
            vgrid={*{6}{draw=none},dotted},
            x unit=3pt,
            y unit chart=20pt,
            time slot format=isodate,
            group label font=\bfseries \Large,
            ]{2021-12-01}{2022-05-29}
            \gantttitlecalendar{year, month=name, week}{1}\\
        
            \ganttgroup[ group/.append ]{2D classical tomography}{2021-12-01}{2022-02-13}\\ 
                \ganttbluebar[name=Implementation 2D]{Implementation in 2D}{2021-12-01}{2022-01-09}\\ 
                \ganttbluebar[name=Implement Baselines]{Implement Baselines for 2D Evaluation}{2022-01-10}{2022-01-23}\\ 
                \ganttbluebar[name=Evaluation 2D]{Evaluation 2D on toy dataset}{2022-01-24}{2022-02-13}\\ 
                \ganttbluebar[name=Related Work]{Related Work}{2022-01-01}{2022-01-10}\\ 
                \ganttgreenbar[name=Honeymoon]{Honeymoon}{2021-12-20}{2022-01-10}
                
            \ganttnewline[thick, black]

            \ganttgroup[ group/.append]{3D cyro-EM}{2022-02-13}{2022-04-10}\\ 
                \ganttbluebar[name=Implementation 3D]{Implementation in 3D}{2022-02-14}{2022-03-06}\\ 
                \ganttbluebar[name=Extend Baselines 3D]{Extend Baselines for 3D Evaluation}{2022-03-07}{2022-03-20}\\ 
                \ganttbluebar[name=Evaluation 3D]{Evaluation 3D on toy dataset}{2022-03-21}{2022-04-10}
            
                \ganttnewline[thick, black]
        
            \ganttgroup[ group/.append ]{Finalization}{2022-04-11}{2022-05-29}\\ 
                \ganttbluebar[name=Final Evaluation Result]{Final Evaluation Results}{2022-04-11}{2022-04-24}\\ 
                \ganttbluebar[name=Thesis]{Writing Thesis}{2022-04-25}{2022-05-29}\\
                \ganttgreenbar[name=Fatherhood]{Birth of first child}{2022-04-20}{2022-04-27}
        
            %Implementing links
            \ganttlink[link bulge=2, link mid=0.5]{Implementation 2D}{Implement Baselines}
            \ganttlink[link bulge=2, link mid=0.5]{Implement Baselines}{Evaluation 2D}
            
            \ganttlink[link bulge=2, link mid=0.5]{Evaluation 2D}{Implementation 3D}
        
            \ganttlink[link bulge=2, link mid=0.5]{Implementation 3D}{Extend Baselines 3D}
            \ganttlink[link bulge=2, link mid=0.5]{Extend Baselines 3D}{Evaluation 3D}
            
            %\ganttlink[link bulge=0.5, link mid=0.01]{Implementation 2D}{Clean-Up}
            %\ganttlink[link bulge=2, link mid=0.3]{Implementation 3D}{Clean-Up}
        
        
            \ganttlink[link bulge=2, link mid=0.5]{Evaluation 2D}{Final Evaluation Result}
            \ganttlink[link bulge=2, link mid=0.5]{Evaluation 3D}{Final Evaluation Result}
            \ganttlink[link bulge=2, link mid=0.5]{Final Evaluation Result}{Thesis}
        
        \end{ganttchart}
        \end{adjustwidth}

\end{landscape}