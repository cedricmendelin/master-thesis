\chapter{Preliminaries and Problem Setup}
\label{sec:preliminariesProblem}


In the following chapter, the problem setup handled by the Master Thesis will be explained.
Further, preliminaries regarding assumptions and other decisions are defined.

\section{Reconstruct problem}
Tomographic reconstruction is a popular inverse problem \cite{tomographicReconstruction}. 
The aim is, to reconstruct an object, from its observed projections.
More formally, the aim is to recover some density function $f$ from overserved samples, taken from the line-integral $p(\cdot)$.

The problem can be defined as a two-dimensional (2-D) problem but also as a three-dimensional (3-D) problem.
In 2-D, also called classical tomography reconstruct problem,  the underlying density function is in two dimensions and the measurements lines lie on a plane.
In 3-D, the density function deals within three dimensions and the measurements are lines with arbitrary orientation in space.


The problem automatically gets harder, if we deal with incomplete datasets (subset of measured lines, limited angle data) but also with noisy observations.
Moreover, the angle $\theta$ of the projections, are not always known. During some observation methods, (cryoEm), the underlying object

\subsection{Notation}

Let start with some notations. First of all, lets define the line integral $p$ of our unknown density function $f$ in the 2-D case:

\begin{equation}
    \begin{aligned}
        p(\theta, s)   &=  R f(\theta, s) \\
        R f(\theta, s) &=  \int_{-\infty}^{\infty} f(x(z), y(z)) dz \\
                       &= \int_{-\infty}^{\infty} f((z \sin \theta + s \cos \theta), (-z \cos \theta + s \sin \theta)) dz \\
    \end{aligned}
\end{equation}

where $p$ is the line integral of the density function $f$, $\theta$ the projection angle and $s$ the distance from the origin.

In the 2-D case, the line integral corresponds to the Radon-Transform \cite{radonTransform}.
With the 2-D Radon transform, we can map the density function $f$ to the sinogram $p$. 
When data of the line integral is presented as a 2-D image, we are speaking from the \textit{sinogram}.
The concept of Radon transform was extended as well and can be applied to higher dimensions as well.

\subsection{Filter Backprojection}
Filter Backprojection (FBP) is a reconstruction method, which allows to solve for $p$. It is equivalent to the inverse of the Radon Transform
and is related to the Fourrier transform. Basically, it maps sinograms of $p$ back to the density function $f$.

The problem with the algorithm is, that it only work for complete data and without noise.

\subsection{Manifold assumption}
In the reconstruct problem, we can apply the Manifold assumption from section \ref{sec:manifoldAssumption}.
Moreover, in the none-noisy case, we can even assume how this Manifold look like.

The manifold, and therefore, a low-dimensional embedding, can be calculated the following:

\begin{enumerate}
    \item Construct the knn-graph from our line integral (sinogram).
    \item Calculate the normalized Graph Laplacian
    \item Extract the second, third (and fourth) smallest eigenvectors
\end{enumerate}

\textbf{todo: Show with Shepp-Logan-phantom that we get unit-cirlce}

The showed example can be extended to 3-D, where the underlying manifold corresponds to the Sphere.
Therefore, we can derive, that our angles have the following property:
In the 2-D case, $\theta \in SO(1)$ and the in 3-D case, $theta \in SO(2)$.

Again, the Circle and Sphere can be computed and for the none-noisy, the underlying Manifold can be seen as known.


\section{Thesis problem}
During the Master Thesis, the reconstruction problem with unknown angles is considered. 
Moreover, the observed samples are considered to be noisy. 

The resulting proposed algorithm should work in the 2-D and 3-D scenario.

The main idea is, to exploit the fact that the underlying Manifold is known (Circle in 2-D and Sphere in 3-D). 

From our noisy observations, we can compute the approximated manifold and compare it with the original manifold.
The comparison between the manifold enables the possibility of a Loss function and Learning in general.




\subsection{Overall notation}

\begin{equation}
    \begin{aligned}
        y_i &= p(\theta, s) + noise \\
        y_i &= A(x, \theta) + noise \\
        p(\theta, s) &= A(x, \theta) \\
    \end{aligned}
\end{equation}

where $y_i$ is the observed sample, $x$ our original object and $A(x, \theta)$ a non-linear operator.



\subsection{2D example}

$a_i \in SO(1)$
$A := N \times N$
$B := N$

$f(x) = \int_{\theta = 0}^{\pi} p(\theta, s) |_{s=x \cdot (- \sin \theta, cos \theta) } d \theta$



\subsection{3D example}

$a_i \in SO(2)$
$A := N \times N \times N$
$B := N \times N$


Simple Case in 
