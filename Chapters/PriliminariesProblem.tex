\chapter{Preliminaries and Problem Setup}
\label{sec:preliminariesProblem}


In the following chapter, the problem setup handled by the Master Thesis will be explained.
Further, preliminaries regarding assumptions and other decisions are explained.

\section{Reconstruct problem}
Tomographic reconstruction is a popular inverse problem. 
The aim is, to reconstruct an object, from its observed projections.

\subsection{Manifold assumption}
\subsection{sinogram}

In computerized tomography (CT) this means to 

\begin{equation}
    y_i = g_i A(\bar{x} ; \alpha_i) + noise,
\end{equation}

where \\
$y_i$ is the observed data \\
$ \bar{x} $  is the original object, \\
$A(\cdot ; \alpha)$ is a non-linear operator parameterized by some value(s) $\alpha$ 
that change from one observation to another, \\
$A(\cdot; \alpha) : \mathbb{R}^{A} \to \mathbb{R}^{B}$
$g$ is a linear (and invertible operator) \\


 

\subsection{2D example}

$a_i \in SO(1)$
$A := N \times N$
$B := N$



\subsection{3D example}

$a_i \in SO(2)$
$A := N \times N \times N$
$B := N \times N$

\cite{radonTransform}


Simple Case in 
