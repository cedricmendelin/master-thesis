\chapter{Foundation}
\label{sec:foundation}


Before digging into the problem setup of the Master Thesis, in the following chapter, a broad foundation
of graphs and Graph Learning will be given. Moreover, some important mathematically concepts will be introduced.
A more detailed explanation to some sections will follow in the Section~\nameref{sec:relatedWork}


\section{Graph Learning}
As already mentioned, Graph Learning is a highly popular research area and got lot of attention in recent years.
It is a new way of applying Machine Learning (ML) and many algorithms emerged from ML.

A lot of real data can be modelled as a graph. The data could have a graph structure, like social networks. 
Or a graph can be artificially constructed with methods like k-nearest neighbours (KNN) or with some other similarity
measure concept.

\subsection{Graph Learning Tasks}
When a graph is available, one can start using Graph Learning algorithms for solving tasks.
Popular tasks are node classification, or link prediction within a graph. One tries to learn from node and edge properties 
as well as the topology of the graph and tries to map the information to a model, which allows prediction or classification.

Another popular task in Graph Learning is community detection, where the aim is to identify cluster of nodes within the input graph.

Further, graphs are highly popular for dimensionality-reduction. In higher dimensions, the euclidean distance is not helpful and therefore,
algorithms for reducing the dimensionality, such that euclidean distance make sense again are needed. 
Graph algorithms provide a helpful tool in such scenarios.

\subsection{Algorithm categories}
\subsubsection{Spectral graph theory}
Spectral graph theory \cite{SpectralGraphTheory} deals with learning properties and characteristics of graphs, in regard to
the graphs eigenvalues and eigenvectors. 

\subsubsection{Graph Deep Learning}
Basically, in Graph Deep Learning, Deep Learning algorithms are extended for the usage with graphs.
After all, a model or some feature will be learned within a neural network, suitable for working with graphs.

\subsubsection{Random walk based methods}
\textbf{TODO}


\section{Graph Foundations}
A graph is defined as  $G = \langle V,E \rangle$, where $V$ is a set of 
vertices (or nodes) and $E$ is a set of edges (or links). Edges are 
defined as a set of tuples $\langle i,j \rangle$, where $i$ and $j$ determine the 
index of the vertices in the graph.

\subsection{Graph Properties}

\subsubsection{Directed vs. undirected vs. weighted}

\subsubsection{Dense and sparse Graph}
A dense graph is a graph, where the number of edges in close to the maximal number of edges.
Contrarily, a sparse graph only consists of a few edges.

\subsection{Node Properties}
\subsection{Edge Properties}
Homophilic and heterophilic are properties of the underlying dataset in link predication. 
A homophilic dataset leads to the tendency to interact with similar nodes. Whereas a heterophilic dataset contrary has the tendency to 
not link, if nodes are similar.

\subsection{Adjacency Matrix}

The adjacency matrix of $G$ is then defined as follows:
\begin{equation}
    A_{ij} =    
    \begin{cases}
        %1  & \text{if } \norm{\biggl y_i - y_j \biggr} < \tau\\
        1  & \text{if } \langle i , j \rangle \in E \\
        0, & \text{otherwise}
    \end{cases}
\end{equation}

\subsubsection{k-hop neighbourhood}
The adjacency matrix $A$ has a nice property. When calculating the $k$-th power of $A^k$ , on calculates the k-hop neighbourhood
of the given matrix. Which can be used for many application.

\subsection{Degree Matrix}
The degree Matrix of $G$ is defined as follows:
\begin{equation}
    D_{ij} =    
    \begin{cases}
        %1  & \text{if } \norm{\biggl y_i - y_j \biggr} < \tau\\
        deg(v_i)  & \text{if } i = j \\
        0, & \text{otherwise}
    \end{cases}
\end{equation}

Where $deg(v_i)$ is the degree of the node, formally the number of incoming edges of node $v_i$.

\subsubsection{Normalization}
When starting calculating with Matrix A, it is sometimes necessary to normalize.
With the degree Matrix $D$ and Adjacency Matrix $A$, we have all information we need.
Mostly, we want to normalize, such that our rows sum to 1.
\begin{equation}
    A_{row-norm} = D^{-1} A
\end{equation}

But we can achieve the same for columns, we just need to swap the two matrices:
\begin{equation}
    A_{col-norm} = A D^{-1}
\end{equation}

And a final, a probably the most useful normalization, is the symmetric normalization:

\begin{equation}
    A_{sym} =  D^{-\frac{1}{2}} A D^{-\frac{1}{2}}
\end{equation}


\subsection{Graph Laplacian}
The graph Laplacian is defined as follows:
\begin{equation}
    L = D - A
\end{equation}


\subsubsection{Normalized Graph Laplacian}

Symmetric normalized: $L_{sym} = I - D^{-\frac{1}{2}} A D^{-\frac{1}{2}}$
Random walk normalized: $L_{rw} = I - D^{-1} A$

\subsubsection{Normalized Graph Laplacian eigen decomposition}

\begin{equation}
    \begin{aligned}
        L_{sym} &= U \Lambda U^T \\
        U &= [u_0, \dots, u_{N-1}] \in R^{N \times N}\\
        \Lambda &= diag \left ( [\lambda_0, \dots, \lambda_{N-1}]  \right ) \in R^{N \times N}
    \end{aligned}
\end{equation}

In this scenario, eigenvectors are also known as \textit{graph Fourier modes}
and eigenvalues are known as the \textit{spectral frequencies}.

Moreover, with the Graph Fourier Transform, we can calculate these values from a symmetric
graph Laplacian.

\subsection{Graph Construction}
\textbf{TODO: KNN}

\section{Graph Denoising}
Data acquired by Real-world observations are often noisy, which can lead to poor 
performance on data analysis tasks. This observed data can already be in the form of a graph,
or a graph can be easily constructed. This resulting graph is what we call
a noisy graph, as it includes the noise from the observation.

Graph denoising is the task to reconstruct the original graph from a noisy one.
Therefore, graph denoising can be seen as a pre-processing step, where noisy data is filtered.

Denoising in general has often to do with averaging 
 and graphs are  a well suited data structure for this task\cite{noneLocalMean}.


\subsection{Noise}
A noisy observation is defined as:
$y_n = y + \eta$

\subsection{Denoising}
When we talk from denoising, we want to reconstruct the true observation 
from a given noisy observation. This reconstruction is done via averaging, which can be performed
locally, by the calculus of variations or in the frequency domain.

\subsection{Noisy Graph}
For every noisy graph, there exists an original graph $G = \langle V,E \rangle$.

The noisy graph can be defined as follows:
\begin{equation}
    \begin{aligned}
        G_{noisy} &= \langle V,E_{noisy} \rangle,  \\ 
        \text{ with }  E_{noisy} &= E \setminus  E^{-} \cup  E^{+}, \\ 
         E^{-} & \subseteq E, \\
         E^{+} \cap E &= \emptyset
    \end{aligned}
\end{equation}

Basically, the noisy graph consists of the same vertices as the original graph. From
the original graphs edges, some are removed (denoted by $E^{-}$) and some new edges are added
(denoted by $E^{+}$).

The adjacency Matrix of $G_{noisy}$ is then defined as follows:
\begin{equation}
    \bar{A}_{ij} =    
    \begin{cases}
        %1  & \text{if } \norm{\biggl y_i - y_j \biggr} < \tau\\
        1  & \text{if } \langle i,j \rangle \in E_{noisy} \\
        0, & \text{otherwise}
    \end{cases}
\end{equation}

The task of graph denoising, can therefore be written as:
\begin{equation}
    \bar{A} \xrightarrow[method]{Graph-denoising} \tilde{A} \approx A
\end{equation}

Where $\bar{A}$ denotes the noisy input graph, $\tilde{A}$ the denoised
 graph and $A$ the original graph.


\subsubsection{Connection to link prediction}
Link prediction is a task in Graph learning. 
The idea is to predict the existence of a link (edge) between two nodes.
The task can be formulated as a missing value estimation task. A model $M_p$ is learned
from a given set of observed edges. The model finally maps links to probabilities:


\begin{equation}
    M_p : E^{\prime} \rightarrow [0,1]
\end{equation}

Where $E^{\prime}$ is the set of potential links.


We define $U$ as the set of all possible vertices of $G$, therefore $E \subseteq U$.
Obviously, one could see Graph denoising as a link prediction problem.

The difference is, that in link prediction, we learn a model from a set of observed links 
$E_{observed} \subseteq E$ and in Graph denoising we learn the model from 
$E_{observed} \subseteq U$. 

On could also say that link prediction problems are a subset of graph denoising problems.



\section{Math Foundation}


\subsection{Power Iterations}
\label{sec:powerIterations}

Power iteration (also called power method) is a iteratively method, 
which approximates the biggest eigenvalue of a diagonalizable matrix A.

The algorithm starts with a random vector $b_0$ or a approximation of the dominant eigenvector.

\begin{equation}
    b_{k+1} = \frac{Ab_k}{||Ab_k||}
\end{equation}

The algorithm not necessarily converge. The algorithm will converge, if A has an eigenvalue strictly grater than its other eigenvalues
and the initial vector $b_0$ has a component in direction of an eigenvector, associated with the dominant eigenvector.

\subsection{Folded spectrum Method}
\label{sec:FoldedSpectrumMethod}
Calculation of eigenvalues and eigenvectors of a given Hamiltonian matrix $H$ 
is a fundamental mathematical problem. Often, we are interested in just the smallest 
values, which can be efficiently computed. But if we are interested in selected values,
this can be hard. $H$ is needed to be diagonalized (bring matrix $H$ into diagonal form) 
which is computationally expensive and for big matrices impossible.

Currently, the best way to solve such problems is the Folded spectrum (FS) \cite{foldedSpectrumMethod} method,
which iteratively solves the problem. During calculation, the eigenvalue spectrum will be folded around a reference 
value $\epsilon$.

\begin{equation}
    v^{t+1} = v^t - \alpha (H - \epsilon I )^2 v^t ,
\end{equation}

with $0 < \alpha < 1$. When $t \rightarrow \infty$, then $v^{\infty}$ should be the 
eigenvector with respect to the reference value $\epsilon$.

\subsection{Embedding}
An embedding $f: X \rightarrow Y$ is defined as a structure-preserving mapping from one domain to another.

In graph theory, a graph embedding in the mapping from the graph $G$
to a surface structure $\Sigma$. 



\section{Manifolds}
A manifold is a topological space, where locally Euclidean distances make sense.
More formally, a $n$-dimensional manifold is a topological space where
each point has e neighbourhood, that is homeomorphic to an subset of a n.dimensional
Euclidean space.

Some example for a 1-D manifold is a line or a circle. 2-D manifolds can already become 
pretty complex and are basically any surfaces like planes, sphere but also the torus,
Klein bottle or others.

\subsection{Manifold assumption}
\label{sec:manifoldAssumption}

The manifold assumption is a popular assumption for high-dimensional datasets.
Even if a given dataset is in high-dimension and consists of many features, one can assume,
that these data points are samples from a low-dimensional manifold, 
which embeds the high-dimensional space.

Therefore, if one can approximate the underlying Manifold, one solved a dimensionality reduction
as we can embed the data points in the low-dimensional manifold space.

There is a complete area of research devotes to this manifold assumption, and basically
we then talk from Manifold Learning. 

\textbf{TODO: write more} \cite{ManifoldLearning}


\subsection{Line integral}
Integral where function is evaluated along a curve

\subsection{SO3}
\subsection{Lie Group}

\subsection{Fourier Transform}
\textit{Fourier Analysis} is the overall field of study, which deals with representing (or approximating) functions as 
sums of trigonometric functions. When the function is defined in such a way, we are talking from the \textit{Fourier Domain}.

\textit{Fourier transform} (FT) is the way of transforming signals to the Fourier Domain, which is popular in ML.
Basically, with the Fourier transform, a signal can be decomposed to a \textit{Fourier series}, which consists of many weighted sinusoids. 


\subsubsection{Fourier-slice theorem}
The Fourier-slice theorem \cite{fourierSliceTheorem} in 2D is defined as follows:

\begin{equation}
    \label{eq:Fourrier-slice}
    F_1 P_1 = S_1 F_2,
\end{equation}

where $F1$ and $F2$ are FTs in 1D and 2D respectively, P1 is a projection operator ($P_1 : 2D \rightarrow 1D$) and $S1$ is a slice operator 
(which slice through the origin).

So basically what it states is, that the 1D FT of a projection is the same, as the 2D FT limiting to the slicing.

\textbf{TODO: make it more formal}


\subsection{Radon Transform}

\section{CT tomography and cryo-EM}

\section{Wasserstein Loss-function}

