\chapter{Imaging methods}
\label{sec:imaging}

In the first chapter, the imaging methods \textit{computed tomography} and 
\textit{cryo-electron microscopy} (cryo-EM) will be introduced. 
Further, their reconstruction is defined in a mathematically way.
Application of cryo-EM is the major motivation for the Master Thesis, 
as the problem is not easy to solve due to dealing with enormous noise.


\section{Computed tomography}
Computed tomography is a well established tomography imaging method.
Using X-ray source, fan shaped beams are produced which scan the imaging object,
resulting in many measurements taken over straight lines \cite{computedTomography}.

\paragraph{Tomography reconstruction:}
Tomographic reconstruction\cite{tomographicReconstruction} is a popular inverse problem \cite{tomographicReconstruction}. 
The aim is to reconstruct an imaged object from observed measurements.
The reconstruction object can be in two-dimension (2D) or in three-dimension (3D). 
In the Master Thesis, the focus will be on 2D case, which is called \textit{classical tomography reconstruction}.

\paragraph{2D tomographic reconstruction:}

Mathematically, the observed measurements can be defined as follows:

\begin{equation}
    \label{eq:2Dreconstruction}
    \begin{aligned}
        y_i[j] &= R(x, \theta_i, s_j) + \eta_i[j] , \text{ with } 1 \leq i \leq N, \text{ and } 1 \leq j \leq M
    \end{aligned}
\end{equation}

where $N$ is number of observations and $M$ the observation dimension.
Then, $x \in L^2(\Omega)$ with $\Omega \subset \mathbb{R}^2 $ is the original object.
Further, $y_i \in \mathbb{R}^M$ is the $i$-th observation with $y_i[j] \in \mathbb{R}$ the $j$-th element of the observation.
$R(\cdot; \theta, s): L^2(\Omega) \to L^2(\tilde{\Omega}) , x \mapsto R(x; \theta,s)$ refers to the Radon Transform
with $\tilde{\Omega} \subset \mathbb{R}$, $\theta$ as the observation angle from the x-axis and $s_j$ as the sampling point.
$\eta$ refers to noise and is defined as $\eta_i[j] \sim \mathcal{N}(0,\sigma^2)$.

\subparagraph{Filter Backprojection:}
Filter Backprojection is a reconstruction method, typically used in classical tomography reconstruction.
It allows to inverse the Radon Transform and is related to the Fourier Transform. 

\section{Cryo-EM}
Cryo-EM is another imaging method, that enables the view of molecules in near-atomic resolution.
In the Master Thesis, only single-particle cryo-EM is considered.
During the imaging process, molecules are frozen in a thin layer of ice, where they are randomly oriented and positioned. 
The freezing process allows to observe molecules in a stable state where they are not moving.
To the contrary, the random orientation and positioning of the molecules makes reconstruction challenging\cite{singleParticleCryoEm}.
With an electron microscope, one can observe two-dimensional tomographic projection images of the molecules in the ice,
which are called \textit{micrograph}. The frozen molecules are fragile and the electron microscope needs to work with
very low power (electron dose), resulting in highly noisy images. The resulting signal-to-noise ration (SNR)
is typically smaller than 1, which indicates that there is more noise than signal\cite{cryoEmMath2}.


\paragraph{3D cryo-EM reconstruction:}
Similar to tomographic reconstruction, there is the cryo-EM reconstruction problem\cite{cryoEmMath}.
It can be seen as a 3D reconstruction problem as the original object $x \in L^2(\Omega)$ to be reconstructed is in 3D,
where now $\Omega \subset \mathbb{R}^3 $ and $\tilde{\Omega} \subset \mathbb{R}^2 $.

Mathematically, the observed measurements can be defined as follows:
\begin{equation}
    \label{eg:cryoEmSimple}
    y_i = \Pi_z ( Rot (x; \theta_i)) + \eta_i, \text{ with } 1 \leq i \leq N
\end{equation}

where $\Pi : L^2(\Omega) \to L^2(\tilde{\Omega}), x \mapsto  \int x(\cdot,\cdot,z) dz$ is the projection operator
and $Rot : L^2(\Omega) \to L^2(\Omega), Rot_\theta(x) = \left((x_1,x_2,x_3) \mapsto x( x_1R^1, x_2R^2, x_3R^3)\right)$ is the rotation operator modelling the rotation during freezing.
Further, $\theta_i = [\theta_i^1, \theta_i^2, \theta_i^3 ] $ where entries $ [\theta_i^1, \theta_i^2, \theta_i^3 \in \mathbb{R}$ and 
$R = [R^1, R^2, R^3] \in SO(3)$. $\eta_i[j,k] \sim \mathcal{N}(0,\sigma^2I)$ corresponds again to the noise of the observation.


As $y_i$ is not observable directly, discretization is needed:
\begin{equation}
    \label{eg:cryoEmSimpleDiscrete}
    \begin{aligned}
        y_i &= \left( \Pi_z ( Rot (x; \theta_i)) + \eta_i\right)(\Delta), \text{ with } 1 \leq i \leq N \\
        y_i[j,k] &= \Pi_z ( Rot(x; \theta_i))_{j,k} + \eta_i[j,k], \text{ with } 1 \leq i \leq N \text{ and } 1 \leq j,k \leq M    
    \end{aligned}
\end{equation}

where $\Delta \subset \tilde{\Omega}^{M^2}$ is the sampling grid and $M$ is the first and second dimension of the sampling grid.


\subparagraph{Extended formula:} 
The equation~\ref{eg:cryoEmSimple} is a simplified version of the cryo-EM reconstruction problem.
First of all, the point spread function (PSF) of the microscope is not taken into account.
Moreover, in cryo-EM the same kind of molecule is observed many times.
Due to structural variety in the molecules, the underlying object $x$ is not the same 
for every observation but can be seen as a random signal from an unknown distribution defined over all possible molecules structures.

The extended version can be defined as follows:
\begin{equation}
    \label{eg:cryoEmExtended}
    y_i = h_i \circ \Pi_z ( Rot (x_i; \theta_i)) + \eta_i, \text{ with } 1 \leq i \leq N
\end{equation}

where $h_i$ is the PSF of the microscope and $\circ$ defines the convolution. 
During the Thesis, the equation~\ref{eq:cryoEmSimpleDiscrete} is used, not the extended version.

\subparagraph{Difference to tomographic reconstruction:}
The two problems are highly related, but the cryo-EM reconstruct is more challenging.
During CT observation, the patient is asked to not move and therefore, the angles of projection is known, whereas
in cryo-EM this information will be lost during the freezing process.
Secondly, the high level of noise makes cryo-EM much more challenging regarding tomographic reconstruction.
The observation 


\section{General from}
As the tomographic reconstruction and the cryo-EM reconstruction are rather similar, 
the aim of the Master Thesis will be to design an algorithm, that can be applied in both scenarios.

Therefore, a general form of the two problem will be defined in the following.
First of all, we define $x \in L^2(\Omega)$, where $L^2$ is the Lebesgue space and $\Omega$
is the sample space $\Omega \subset \mathbb{R}^D$, where $D$ is the dimension of the sample space. 
Further we define $\tilde{\Omega} \subset \mathbb{R}^{D-1}$.

\begin{equation}
    \begin{aligned}
        y_i &= \left( A(x, \theta_i) + \eta_i \right) (\Delta)\\
    \end{aligned}
\end{equation}

where $y_i \in \tilde{\Omega}^M$ is the observed sample, $M$ the sample dimension, $x \in L^2(\Omega)$ our original object, $A$ a non-linear operator 
$A: L^2(\Omega) \to L^2(\tilde{\Omega}), x \mapsto A(x; \theta)$ and
$\eta ~ \mathcal{N}(O, \sigma^2 I)$ gaussian noise. $\Delta \subset \tilde{Omega}^{M^2}$ is a term for discretization.

\paragraph{Classical tomography reconstruction:}

For classical tomography, the parameters can be defined with $D=2$ and $\theta \in \mathbb{R}^1$.
Further, $A(\cdot)$ can be defined as the Radon transform.
A distance measure between samples can be set up by using the l2-norm $\norm{y_i - y_j}$.

\paragraph{Cryo-Em reconstruction:}
For cryo-EM, the parameters can be defined with $D=3$ and $\theta \in \mathbb{R}^3$.
Further, $A(\cdot)$ can be defined as $\Pi_z \left( Rot(x; \theta) \right)$ 
where $Rot$ is the 3D rotation and $\Pi_z$ the tomographic projection.

As the samples are drawn with some random 3D rotation and then will be projected, it can 
happen that two samples are equivalent up to an 2D rotation. 
Consider a first example $y_1$, which has no 3D rotation at all and 
a second sample $y_2$ with a rotation only in in x-y plane by 45°.
The two samples have a defined in-plane rotation $g$, such that $g y_1 = y_2$.
Therefore, in our distance measure we add this term of in-plan rotation: $min_{g \in SO(1)}\norm{g * y_i - y_j}$, 
which is inspired by the work of \cite{multiDiffusionMaps}. 


Transition: noise is the main problem in single particle cryo-EM. In the thesis we will look at way remove it. 
In the next chapter, we will see that it has strong connections with Graphs.

\begin{tcolorbox}[colback=red!5!white,colframe=red!75!black]
    Cryo-EM reconstruction needs to define algorithms, which works well in high noise regimes.
    During Master Thesis only on single-particle cryo-EM is considered, so speaking from cryo-EM 
    it refers to single-particle cryo-EM.
\end{tcolorbox}
