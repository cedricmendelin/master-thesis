\chapter{Assessment criteria}
Written report including: 
\begin{itemize}
    \item Contents of the Master's Thesis project
    \item Project plan
    \item summary of relevant related work
\end{itemize}

\chapter{Questions}


\textbf{Trainable Laplacian with folded FSM}
\begin{itemize}
    \item Noisy projection samples? Yes!
    \item What is the underlying problem to solve?
    \item Other graph creation mechanism to consider.
    \item Similarity measure?
    \item Wasserstein Loss function
\end{itemize}

\textbf{Questions GCN:}
\begin{itemize}
    \item During feature propagation, only node features are considered.
    \item Spectral Analysis Chapter in SGCN
\end{itemize}

\textbf{Questions Random projection:}
\begin{itemize}
    \item Assumption about uniform(0, 2$\pi$) and equally spaced $t$?
    \item First-order Chebyshev
\end{itemize}

\textbf{Questions Walk Pooling}
\begin{itemize}
    \item What is vanilla graph classification?
    \item Subgraph classification?
\end{itemize}


\textbf{Next Steps:}
\begin{itemize}
    \item Familiarize with CryoEm
    \item Do some coding
    \item Read paper regarding Manifold Learning
\end{itemize}

\subsection{Walk Pooling}

Homophilic and heterophilic are properties of the underlying dataset in link predication. 
A homophilic dataset leads to the tendency to interact with similar nodes. Whereas a heterophilic dataset contrary has the tendency to 
not link, if nodes are similar.

\citet{walkPooling} proposed a new way of link prediction, which works
with a random-walk-based pooling method called WalkPool.

Based on the assumption, that link presence can be predicted 
only based on the information on its neighbours within a small radius $k$,
they calculate subgraphs and extract information from these subgraphs.

\subsubsection{Feature extraction}
Based on the adjacency matrix and some node attributes, 
feature will be extracted with a GNN $f_{\theta}$

\begin{equation}
    Z = f_{\theta} (A,X)
\end{equation}

This step could be achieved by the earlier introduced GCN.

\subsubsection{Subgraph classification}

For candidate link $E^c$, the k-hop enclosing subgraphs
 will be constructed, which allows the transformation
from a link predication problem to a graph classification problem.


From these subgraphs, so called random-walk profiles are calculated:

\subsubsection{Random-walk profiles}

First of all, the node correlation is calculated with two multilayer perceptrons (MLP)

\begin{equation}
    w_{x,y} = \frac{Q_{\theta}(z_x)^T K_{\theta}(z_y)}{\sqrt{F^{\prime\prime}}}
\end{equation}

Then, from all neighbouring nodes ($N(x)$), probabilities are calculated:
\begin{equation}
    p_{x,y} = 
    \begin{cases}
            %1  & \text{if } \norm{\biggl y_i - y_j \biggr} < \tau\\
            [softmax(w_{x,z})_{z \in N(x)}]_y  & \text{if } \langle x , y \rangle \in E \\
            0, & \text{otherwise}
    \end{cases}
\end{equation}

From these calculated probabilities $P$ and its powers, we can derive information for 
graph classification:

Node level features $node^{\tau}$ describe loop structures around the candidate link.
Further, link features $link^{\tau}$ give a probability, about a random walk with length
$\tau$ from the two nodes ( ending in a loop). And finally, graph features $graph^{\tau}$ are related to the total
probability of length $\tau$ loops.

\textbf{TODO:}
As we do not know about if the candidate link is present of not, we consider the subgraphs,
consisting of the candidate link and without the candidate link.


We finally can describe WalkPool as:

\begin{equation}
    WP_{\theta}(G, Z) =w_{1,2}(node^{\tau, +}, node^{\tau, -}, link^{\tau, +}, link^{\tau, -}, \delta graph^{\tau})^{\tau_c}_{\tau = 2}
\end{equation}

