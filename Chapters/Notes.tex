\chapter{Assessment criteria}
Written report including: 
\begin{itemize}
    \item Contents of the Master's Thesis project
    \item Project plan
    \item summary of relevant related work
\end{itemize}

\chapter{Questions}


\textbf{Organization}
\begin{itemize}
    \item Presentation, date and content?
    \item Contract of Thesis
    \item Project plan
    \item How detailed Related work?
    \item Send Reminder to Ivan (end of the week)?
\end{itemize}

\textbf{Feedback questions:}
\begin{itemize}
    \item Finally, the goal of the master thesis is to produce methods to estimate G_0 based on G, is it?
    \item Motivation form cryo-EM for "Graph Denoising"
    
    
    \item Graph Laplacian and machine learning: this is one on the main idea to explore in the thesis. 
        Graph Laplacian is a powerful tool, but relies on the noisy adjacency matrix. 
        Can we learn the adjacency matrix such that the output of the Graph Laplacian is what we expected (link with spectrum folded).
    \item ML powermethod to Graph Laplacian
        Compute Evec of Laplacian which works for ML?
    \item Aim expolit Connection between the GNN and Graph Laplacian.
    \item Mathematical analysis of the link between Graph neural network and Graph Laplacian, as in the paper "Simplifying graph convolutional networks". Another strong idea to explore.
    \item FSM may be used at some point, but maybe not.
    \item link with signal processing (paragraph currently called "Manifold assumption") -> common in signal processing.
    \item 3.3: discretization of problem
    \item Define SNR Ratio we want to work with. (V Shape should be solveable)

\end{itemize}
